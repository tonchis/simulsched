\section{Lottery scheduling}
Lottery scheduling es un algoritmo de planificaci\'on probabilistico para los procesos de un sistema operativo. A cada proceso se les asigna un n\'umero de billetes de loter\'ia, y se elije un boleto al azar para seleccionar el siguiente proceso. La distribución de los billetes no tiene por qu\'e ser uniforme, darle más billetes a un proceso le ofrece una oportunidad mayor de salir seleccionado.

\subsection{Implementaci\'on propuesta}
A la hora de implementar este algoritmo, recurrimos al paper de Waldspurger y Weihl, \textit{Lottery scheduling: Flexible proportional - share resource management} el cual explica los detalles del algoritmo. 

En dicho paper, adem\'as del sistema de tickets se propone un sistema de monedas o \textit{currencies} las cuales respaldan los tickets. El sistema cuenta con una moneda base que respalda una serie de tickets repartidos entre los usuarios que lanzan los procesos. A su vez, cada usuario emite una moneda para respaldar los tickets que reciben sus procesos. Lo mismo sucede con los procesos y sus \textit{threads}. Como en el simulador provisto por la c\'atedra no existen los usuarios ni los \textit{threads}, implementamos \'unicamente el sistema de tickets base.

Cuando un nuevo proceso se inicia, este recibe un ticket para el pr\'oximo sorteo. Como todas las tareas reciben un ticket, todas tienen la misma probabilidad de ser elegidas.

Por otro lado, cuando un proceso consume una porci\'on \textit{f} de su \textit{quantum} y luego se bloquea, este es desalojado del procesador y compensado con $\frac{quantum}{quantum - f}$ cantidad de tickets para el pr\'oximo sorteo. 