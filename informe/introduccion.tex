\section{Introducción}

\begin{quote}

\textit{El planificador (o scheduler en inglés) es un componente funcional muy importante de los sistemas operativos multitarea y multiproceso, y es esencial en los sistemas operativos de tiempo real. Su función consiste en repartir el tiempo disponible de un microprocesador entre todos los procesos que están disponibles para su ejecución.\footnote{Wikipedia: La enciclopedia libre}}

\end{quote}

Existen dos tipos de algoritmos de \emph{scheduling}, cooperativo y con desalojo. Los algoritmos de tipo cooperativo permiten la ejecución de una tarea hasta que ésta finalice. Los del tipo con desalojo asignan un tiempo de ejecución a cada tarea después del cual se desaloja (si no llegó a su fin) y se ejecuta otra tarea por la misma cantidad de tiempo y así hasta que cada tarea termine. 



