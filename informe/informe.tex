\documentclass[a4paper,spanish]{article}
\usepackage[spanish]{babel}
\usepackage[utf8]{inputenc}
\usepackage{caratula}
\usepackage{amsmath, amscd, amssymb, amsthm, latexsym, verbatim}
\usepackage{graphicx, graphics, caption}
\usepackage{fancyhdr}
\usepackage{float, algorithmic}
\usepackage{hyperref}
\usepackage{algorithm}

%probando para los margenes
\usepackage[top=3cm, bottom=3cm, left=1cm, right=1cm]{geometry}

% configuro el paquete de algoritmos
\floatname{algorithm}{Algoritmo}

\makeatletter
\newcounter{algorithmic}
\let\ORIG@algorithmic\algorithmic
\def\algorithmic{\stepcounter{algorithmic}\ORIG@algorithmic}
\def\theHALC@line{\thealgorithmic-\theALC@line}
\def\theHALC@rem{\thealgorithmic-\theALC@rem}
\makeatother

% encabezados
\newcommand{\norma}[1]{\left|\left|#1\right|\right|}
\parskip=1ex
\pagestyle{fancy}
\pagenumbering{arabic}
%\fancyhf{}
\renewcommand{\headrulewidth}{0.02 cm}
\renewcommand{\footrulewidth}{0 cm}
\fancyhead[L]{Trabajo Pr\'actico N$^{\circ}$ 1}
\fancyhead[R]{Sistemas operativos}
\def\septad{\rule{16 cm}{.2 mm}}

% defino un environment propio para las ecuaciones
%\newenvironment{ecuacion}
%	{\begin{equation} \begin{aligned}}
%	{\end{aligned} \end{equation}}
%	
%\newenvironment{ecuacion*}
%	{\begin{equation*} \begin{aligned}}
%	{\end{aligned} \end{equation*}}

% comienzo el documento
\begin{document}

	\materia{Sistemas operativos}

\titulo{Trabajo Práctico 1}

%\grupo{Grupo Beto Quantró}

\integrante{Ramiro Camino}{264/06}{ramaramarama@gmail.com}

\integrante{Gloria Josefina Diodati}{285/05}{josefa84@gmail.com}

\integrante{José Luis Escalante Catácora}{822/06}{joeloui@yahoo.com.ar}

\maketitle

	
	\tableofcontents

	\section{Introducción}

\begin{quote}

\textit{El planificador (o scheduler en inglés) es un componente funcional muy importante de los sistemas operativos multitarea y multiproceso, y es esencial en los sistemas operativos de tiempo real. Su función consiste en repartir el tiempo disponible de un microprocesador entre todos los procesos que están disponibles para su ejecución.\footnote{Wikipedia: La enciclopedia libre}}

\end{quote}

Existen dos tipos de algoritmos de \emph{scheduling}, cooperativo y con desalojo. Los algoritmos de tipo cooperativo permiten la ejecución de una tarea hasta que ésta finalice. Los del tipo con desalojo asignan un tiempo de ejecución a cada tarea después del cual se desaloja (si no llegó a su fin) y se ejecuta otra tarea por la misma cantidad de tiempo y así hasta que cada tarea termine. 




	\section{Round Robin scheduling}

\begin{quote}
	\textit{Round robin es un método para seleccionar todos los elementos en un grupo de manera equitativa y en un orden racional, normalmente comenzando por el primer elemento de la lista hasta llegar al último y empezando de nuevo desde el primer elemento.} 
	
	\textit{El nombre del algoritmo viene del principio de Round-Roubin conocido de otros campos, donde cada persona toma una parte de un algo compartido en cantidades parejas.}\footnote{Fuente: Wikipedia: La enciclopedia libre.}
	
\end{quote}
	
	El concepto del algoritmo se basa en la definición de un intervalo de tiempo llamado \textit{quantum} cuya duración es variable. Éste suele implementarse mediante un temporizador que genera una interrupción cuando se agota el tiempo.
	
	Las tareas se disponen en una cola simulando una estuctura circular; la organización de la misma es FIFO. Si una tarea agota su tiempo de procesamiento antes de finalizar el \textit{quantum}, ésta es desalojada y en su lugar se asigna otra tarea.
	
	Vale destacar que este algoritmo, si bien tiene un tiempo de espera grande, garantiza un reparto equitativo del procesador entre todas las tareas lo cual lo hace libre de inanición.

\subsection{Implementación propuesta}

Nuestra implementación del scheduler contiene tres métodos: \textit{load(pid), unblock(pid) y tick(motivo)}, además de una cola para los procesos:

\begin{itemize}
	\item El método \textit{load(pid)} introduce una tarea con identificador \textit{pid} en la cola de tareas.
	
	\item El método \textit{unblock(pid)} avisa al \textit{scheduler} cuando la tarea con identificador \textit{pid} deja de estar bloqueada introduciéndola en la cola de tareas.
	
	\item El método \textit{tick(motivo)} se ejecuta por cada tick del reloj de la máquina el simulador. El parámetro \textbf{motivo} indica lo que ocurrió con la tarea que ocupaba el CPU el ciclo de reloj anterior:
	
	\begin{itemize}
		\item Si el \textbf{motivo} es \textit{block o exit}: se reinicializa el \textit{quantum} y se saca la tarea de la cola. Cuando esto sucede y la cola queda vacía, se devuelve \verb|idle_task| ; de lo contrario el \textit{pid} de la tarea que ocupará el próximo ciclo de reloj.

		\item Si el \textbf{motivo} es \textit{tick}: si la tarea no era \verb|idle_task| y su \textit{quantum} finalizó, se devuelve el \textit{pid} de la tarea siguiente en la cola. Si todavía le queda tiempo para procesar, el \textit{pid} de la tarea que se procesó es el mismo que se retorna.
		
		Si la tarea era \verb|idle_task| y la cola está vacía, entonces continúa ejecutándose la misma tarea; de lo contrario se devuelve el \textit{pid} de la primer tarea de la cola y se reinicializa el \textit{quantum}.
	\end{itemize}

\end{itemize}

\subsection{Diagrama de estados}

\begin{center}
\includegraphics[scale=0.5]{estados.png}
\end{center}
 
En este diagrama de estados podemos apreciar el desalojo de tareas así como 

\subsection{Análisis del algoritmo}

Al realizar varias pruebas pudimos comprobar que a medida que aumentamos el \textit{quantum} haciéndolo proporcional al total de \textit{ticks} de las tareas, el tiempo necesario para que todas finalicen disminuye. Además podemos ver que el \textit{throughput} también mejora: a medida que el \textit{quantum} aumenta la proporción de tareas finalizadas por intervalo es crece.

Esto es lo que podemos apreciar en los gráficos siguientes: las tareas, de tipo \verb|TaskBatch|, duran 10 \textit{ticks} y la cantidad de \textit{blocks} aumenta de 0 a 9. 

\begin{figure}[H]
\centering
\includegraphics[scale=0.4]{./graficos/out_batch_fijo1.png}
\caption{Algoritmo Round robin con \textit{quantum} 1}
\end{figure}  

Como podemos apreciar este caso las tareas tardan alrededor de 200 \textit{ticks} de reloj en terminar. En cuanto al \textit{throughput} podemos ver que es bastante malo ya que todas las tareas terminan en un mismo intervalo, el último. 

 
\begin{figure}[H]
\centering
\includegraphics[scale=0.4]{./graficos/out_batch_fijo9.png}
\caption{Algoritmo Round robin con \textit{quantum} 9} 
\end{figure}


En este caso las tareas tardan alrededor de 160 \textit{ticks} de reloj en terminar y con respecto al \textit{throughput} vemos que mejora notablemente ya que la cantidad de tareas que finalizan por intervalo es mas uniforme. 



	\section{Lottery scheduling}
Lottery scheduling es un algoritmo de planificaci\'on probabilistico para los procesos de un sistema operativo. A cada proceso se les asigna un n\'umero de billetes de loter\'ia, y se elije un boleto al azar para seleccionar el siguiente proceso. La distribución de los billetes no tiene por qu\'e ser uniforme, darle más billetes a un proceso le ofrece una oportunidad mayor de salir seleccionado.

\subsection{Implementaci\'on propuesta}
A la hora de implementar este algoritmo, recurrimos al paper de Waldspurger y Weihl, \textit{Lottery scheduling: Flexible proportional - share resource management} el cual explica los detalles del algoritmo. 

En dicho paper, adem\'as del sistema de tickets se propone un sistema de monedas o \textit{currencies} las cuales respaldan los tickets. El sistema cuenta con una moneda base que respalda una serie de tickets repartidos entre los usuarios que lanzan los procesos. A su vez, cada usuario emite una moneda para respaldar los tickets que reciben sus procesos. Lo mismo sucede con los procesos y sus \textit{threads}. Como en el simulador provisto por la c\'atedra no existen los usuarios ni los \textit{threads}, implementamos \'unicamente el sistema de tickets base.

Cuando un nuevo proceso se inicia, este recibe un ticket para el pr\'oximo sorteo. Como todas las tareas reciben un ticket, todas tienen la misma probabilidad de ser elegidas.

Por otro lado, cuando un proceso consume una porci\'on \textit{f} de su \textit{quantum} y luego se bloquea, este es desalojado del procesador y compensado con $\frac{quantum}{quantum - f}$ cantidad de tickets para el pr\'oximo sorteo. 
	\section{Conclusión}

Para la realización de este trabajo se nos pidió que implementáramos dos algoritmos de \textit{scheduling} así también como varios tipos de tareas.
Luego de hacerlo podemos decir que en general hay muchas variables a tener en cuenta y mejorar pero, a veces, estas se contradicen y no es posible mejorar algunas sin empeorar otras.

En el algoritmo de \textit{Round Robin scheduling} logramos evitar el problema de inanición mediante la asignación de un \textit{quantum} pero todas las tareas son consideradas iguales. Si el \textit{quantum} es muy largo, puede parecer que el sistema no responde. Esto se puede reflejar, por ejemplo, en que si estoy utilizando el juego ``Buscaminas'' y el reproductor de música, podría escuchar el sonido entrecortado.

Si el \textit{quantum} es muy corto, éste es consumido en su mayor parte por \textit{scheduling} y cambios de contexto, lo cual el sistema pasa gran parte de su tiempo sin hacer trabajo de verdad.

En el algoritmo de \textit{Lottery scheduling} se trata de solucionar los inconvenientes anteriormente mencionados mediante la asignación de prioridades a las tareas, sin embargo, dado que la cantidad de tickets que se utiliza para seleccionar la siguiente tarea a procesar proviene de un número aleatorio, puede suceder que alguna tarea permanezca sin procesar por mucho tiempo.

Vale destacar que este algoritmo, al igual que \textit{Round robin} también soluciona el problema de inanición.

Finalmente no podemos decir que alguno de los dos algoritmos de \textit{scheduling} implementados sea mejor que el otro. Todo depende del contexto en el que se esté.

 
	%\begin{thebibliography}{9}

\bibitem {} Thomas H. Cormen, Charles E. Leiserson, Ronald L. Rivest, y Clifford Stein. Introduction to Algorithms, Second Edition. MIT Press and McGraw-Hill, 2001.

\bibitem {} Xumari, G.L. Introduction to dynamic programming. Wilwy $\&$ Sons Inc., New York. 1967.

\end{thebibliography}

	%palmitos, champignones a la provenzal 2, panceta y ciruela,
	%provolone 3, queso al oreganato 2, pollo 1
	
	%panceta? anana?

\end{document}