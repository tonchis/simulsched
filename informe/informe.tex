\documentclass[a4paper,spanish]{article}
\usepackage[spanish]{babel}
\usepackage[utf8]{inputenc}
\usepackage{caratula}
\usepackage{amsmath, amscd, amssymb, amsthm, latexsym, verbatim}
\usepackage{graphicx, graphics, caption}
\usepackage{fancyhdr}
\usepackage{float, algorithmic}
\usepackage{hyperref}
\usepackage{algorithm}

%probando para los margenes
\usepackage[top= 3cm, bottom= 3cm , left= 2.5cm, right= 2.5 cm]{geometry}

% configuro el paquete de algoritmos
\floatname{algorithm}{Algoritmo}

\makeatletter
\newcounter{algorithmic}
\let\ORIG@algorithmic\algorithmic
\def\algorithmic{\stepcounter{algorithmic}\ORIG@algorithmic}
\def\theHALC@line{\thealgorithmic-\theALC@line}
\def\theHALC@rem{\thealgorithmic-\theALC@rem}
\makeatother

% encabezados
\newcommand{\norma}[1]{\left|\left|#1\right|\right|}
\parskip=1ex
\pagestyle{fancy}
\pagenumbering{arabic}
%\fancyhf{}
\renewcommand{\headrulewidth}{0.02 cm}
\renewcommand{\footrulewidth}{0 cm}
\fancyhead[L]{Trabajo Pr\'actico N$^{\circ}$ 1}
\fancyhead[R]{Sistemas operativos}
\def\septad{\rule{16 cm}{.2 mm}}

% defino un environment propio para las ecuaciones
%\newenvironment{ecuacion}
%	{\begin{equation} \begin{aligned}}
%	{\end{aligned} \end{equation}}
%	
%\newenvironment{ecuacion*}
%	{\begin{equation*} \begin{aligned}}
%	{\end{aligned} \end{equation*}}

% comienzo el documento
\begin{document}

	% **************************************************************************
%
%  Package 'caratula', version 0.2 (para componer caratulas de TPs del DC).
%
%  En caso de dudas, problemas o sugerencias sobre este package escribir a
%  Nico Rosner (nrosner arroba dc.uba.ar).
%
% **************************************************************************



% ----- Informacion sobre el package para el sistema -----------------------

\NeedsTeXFormat{LaTeX2e}
\ProvidesPackage{caratula}[2003/4/13 v0.1 Para componer caratulas de TPs del DC]


% ----- Imprimir un mensajito al procesar un .tex que use este package -----

\typeout{Cargando package 'caratula' v0.2 (21/4/2003)}


% ----- Algunas variables --------------------------------------------------

\let\Materia\relax
\let\Submateria\relax
\let\Titulo\relax
\let\Subtitulo\relax
\let\Grupo\relax


% ----- Comandos para que el usuario defina las variables ------------------

\def\materia#1{\def\Materia{#1}}
\def\submateria#1{\def\Submateria{#1}}
\def\titulo#1{\def\Titulo{#1}}
\def\subtitulo#1{\def\Subtitulo{#1}}
\def\grupo#1{\def\Grupo{#1}}


% ----- Token list para los integrantes ------------------------------------

\newtoks\intlist\intlist={}


% ----- Comando para que el usuario agregue integrantes

\def\integrante#1#2#3{\intlist=\expandafter{\the\intlist
    \rule{0pt}{1.2em}#1&#2&\tt #3\\[0.2em]}}


% ----- Macro para generar la tabla de integrantes -------------------------

\def\tablaints{%
    \begin{tabular}{|l@{\hspace{4ex}}c@{\hspace{4ex}}l|}
        \hline
        \rule{0pt}{1.2em}Integrante & LU & Correo electr\'onico\\[0.2em]
        \hline
        \the\intlist
        \hline
    \end{tabular}}

% ----- Macro para generar la parte reservada para la c�tedra -------------------------

\def\tablacatedra{%
    \\
    \textbf{Reservado para la c\'atedra}\par\bigskip
    \begin{tabular}{|c|c|c|}
        \hline
        \rule{0pt}{1.2em}Instancia & Docente & Nota\\[0.2em]
        \hline
        \rule{0pt}{1.2em}Primera entrega & \phantom{mmmmmmmmmmmmmmmmmm} & \phantom{mmmmmm} \\
        \hline
        \rule{0pt}{1.2em}Segunda entrega & & \\
        \hline
    \end{tabular}}

% ----- Codigo para manejo de errores --------------------------------------

\def\se{\let\ifsetuperror\iftrue}
\def\ifsetuperror{%
    \let\ifsetuperror\iffalse
    \ifx\Materia\relax\se\errhelp={Te olvidaste de proveer una \materia{}.}\fi
    \ifx\Titulo\relax\se\errhelp={Te olvidaste de proveer un \titulo{}.}\fi
    \edef\mlist{\the\intlist}\ifx\mlist\empty\se%
    \errhelp={Tenes que proveer al menos un \integrante{nombre}{lu}{email}.}\fi
    \expandafter\ifsetuperror}


% ----- Reemplazamos el comando \maketitle de LaTeX con el nuestro ---------

\def\maketitle{%
    \ifsetuperror\errmessage{Faltan datos de la caratula! Ingresar 'h' para mas informacion.}\fi
    \thispagestyle{empty}
    \begin{center}
    \vspace*{\stretch{2}}
    {\LARGE\textbf{\Materia}}\\[1em]
    \ifx\Submateria\relax\else{\Large \Submateria}\\[0.5em]\fi
    \par\vspace{\stretch{1}}
    {\large Departamento de Computaci\'on}\\[0.5em]
    {\large Facultad de Ciencias Exactas y Naturales}\\[0.5em]
    {\large Universidad de Buenos Aires}
    \par\vspace{\stretch{3}}
    {\Large \textbf{\Titulo}}\\[0.8em]
    {\Large \Subtitulo}
    \par\vspace{\stretch{3}}
    \ifx\Grupo\relax\else\textbf{\Grupo}\par\bigskip\fi
    \tablaints
    \vspace*{\stretch{3}}
    \medskip
    \tablacatedra
    \end{center}
    \vspace*{\stretch{3}}
    \newpage
    }

	
	\tableofcontents

	\section{Introducción}


	%\include{bynuno}
	%\include{byninf}
	%\include{roberts}
	%\include{prewitt}
	%\include{sobel}
	%\include{freichen}
	\begin{thebibliography}{9}

\bibitem {} Thomas H. Cormen, Charles E. Leiserson, Ronald L. Rivest, y Clifford Stein. Introduction to Algorithms, Second Edition. MIT Press and McGraw-Hill, 2001.

\bibitem {} Xumari, G.L. Introduction to dynamic programming. Wilwy $\&$ Sons Inc., New York. 1967.

\end{thebibliography}

\end{document}